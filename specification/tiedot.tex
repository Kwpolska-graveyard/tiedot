\documentclass[a4paper,english]{paper}
\usepackage{fixltx2e}
\usepackage{lmodern}
\usepackage{tgheros} %helvet}
\usepackage{DejaVuSansMono}
\usepackage[T1]{fontenc}
\usepackage[utf8]{inputenc}
\setcounter{secnumdepth}{5}
\setcounter{tocdepth}{5}
\usepackage{color}
\usepackage{enumerate}
\usepackage{varioref}
\usepackage{setspace}
\PassOptionsToPackage{normalem}{ulem}
\usepackage{ulem}
\usepackage{hyphenat}
\usepackage[parfill]{parskip}
\setlength{\parskip}{\smallskipamount}
\setlength{\parindent}{0pt}
\usepackage{setspace}
\usepackage{textcomp}
\usepackage{xfrac}
\usepackage{graphicx}
\usepackage{booktabs}
\usepackage{array}
\usepackage{paralist}
\usepackage{verbatim}
\usepackage{subfig}
\usepackage{mathtools}
\usepackage{multirow}
\usepackage{tabularx}
\usepackage[vario]{fancyref}
\def\hypdate{\kern.1em-\kern.1em}
\def\ndash{\ts--\hskip.25em}
\def\mdash{\ts---\hskip.25em}
\onehalfspacing
\nonfrenchspacing
\makeatletter
\pdfpageheight\paperheight
\pdfpagewidth\paperwidth

\newcommand{\noun}[1]{\textsc{#1}}
\providecommand{\\}{\\}
\date{}
\makeatother
\setlength{\topmargin}{-0.5in}
\setlength{\textheight}{10in}
\setlength{\oddsidemargin}{.125in}
\setlength{\textwidth}{6.25in}
\newcommand{\version}{0.1.0}
\usepackage{babel}
\begin{document}
\title{Tiedot Specification}
\subtitle{version \version}
\author{\emph{Chris }Kwpolska\emph{ Warrick}}
\date{2012-11-25T20:00:00Z}

\maketitle
\pagenumbering{roman}

\textsf{The key words ``MUST'', ``MUST NOT'', ``REQUIRED'', ``SHALL'', ``SHALL
    NOT'', ``SHOULD'', ``SHOULD NOT'', ``RECOMMENDED'',  ``MAY'', and ``OPTIONAL''
in this document are to be interpreted as described in RFC 2119.}

\tableofcontents


%%% PART I. Rationale %%%


\pagebreak
\pagenumbering{arabic}
\setcounter{page}{1}
\part{Rationale}
\label{Rationale}

The Tiedot project is a data storage system by Chris \emph{Kwpolska} Warrick,
built on top of python and the following elements:
\begin{enumerate}
    \item \textbf{The Tiedot Rationale}, this paragraph, which describes the need for
        this whole thing;
    \item \textbf{Engine}, including:
        \begin{enumerate}[(1)]
            \item \textbf{Python OOP Engine}, providing specific classes for every
                Element and its needs;
            \item \textbf{TDP}, for storing the data;
            \item \textbf{TDR}, alias references for the data;
            \item \textbf{Authentication}, used for access control.
        \end{enumerate}
    \item \textbf{Network Access}, used to  access data from the network, in the
        following ways (\emph{Actions}):
        \begin{enumerate}[(1)]
            \item \textbf{Sync}, syncing data between the client and the
                server, allowing to use the data by the client without a
                network connection;
            \item \textbf{Live Network Access} (\textsc{lna}), used to access
                the data one-by-one, and send them in the same way;
            \item \textbf{Web Client}, used to access data through a web browser.
        \end{enumerate}
        Additionally, all data access must happen through encrypted protocols.
\end{enumerate}

\pagebreak


%%% PART II. Engine %%%


\part{Engine}
\label{chap:Engine}
\section{Python OOP Engine}
\label{sec:Python OOP Engine}

The main engine of the Tiedot is the Python programming language.  Its
technologies and awesomeness are used to store, create and make use of the
objects.  Main module used is the \texttt{pickle} module, explained further in
\fref{sec:TiedotDP} (\textsc{tdp}---Tiedot Data Pickles).

Since 2012-12-01, Python must be in version 2.7.3.  Otherwise, the Sync
protocol will not work.

\section{TDP}
\label{sec:TDP}

Tiedot Data Pickles are the files storing all objects of the System.  They are the
standard Python pickles, with protocol version 2.  They contain the
\texttt{Object()}s and other objects used by the system (authentication data,
\textsc{tdr}s)

\section{TDR}
\label{sec:TDR}

TDR is an additional, optional reference to an object.  Created by something
as easy as \texttt{TDR = full.object.path}, they are providing an easy way to
refer to objects that are referred to a lot.  Although the method of creation
is very easy, to ensure safe performance of all authorized activities, they
should be modified using the \texttt{addtdr()} and \texttt{rmtdr()}
functions and with \texttt{obj.tdr} set to the \textsc{tdr}.  Otherwise, they will
not be remembered for future sessions.

The suggested length of a TDR is \textbf{4}, although it doesn't matter to the
Tiedot.  This is a suggestion by a human, for humans.

\section{Authentication}
\label{sec:Authentication}

All authentication stuff is done by \texttt{tiedot.AUTH}, an instance of
\texttt{tiedot.auth.Auth}.  It should be used with credential providers, eg.
\texttt{tiedot.ui.cli.auth}.

Account modification actions require synchronization with the sync server, using the \texttt{USERMOD} action.
\pagebreak


%%% PART III. Network Access %%%


\part{Network Access}
\label{chap:Network Access}
\setcounter{section}{0}
Tiedot has multiple Network Access features.  They are using Twisted, Python and
other solutions to provide a consistent interface.

\section{\textsc{BP}}
\label{sec:BP}
BP stands for \emph{Base64-Pickle}.  It is a method of transporting data.
Pickle protocol 2 applies.

\section{Data Dict}
\label{sec:Data Dict}
The Data Dict is a dict of file SHA1 sums (\texttt{s}) and timestamps
(\texttt{t}), transported as a BP.

\section{Data Comparison Procedure}
\begin{tabular}{|ll|}
    \hline
    \textbf{ITMS} & All items. \\
    \textbf{HAVE} & Items to be sent to the client. \\
    \textbf{WANT} & Items to be received from the client. \\
    \textbf{NONE} & Items that are in sync on both sides. \\\hline
\end{tabular}

~

\begin{enumerate}
    \item Compare items.
        \begin{enumerate}
            \item \(=\) Continue.
            \item \(\neq\) Add missing items to dicts with sha1sums and timestamps set to 0.
        \end{enumerate}
    \item Iterate over items.
        \begin{enumerate}
            \item Add the item to \(\rightarrow\)\textbf{ITMS}.
            \item Compare checksums.
                \begin{enumerate}
                    \item \(=\text{ }\rightarrow\)\textbf{NONE}.
                    \item \(\neq\) Compare timestamps.
                        \begin{enumerate}
                            \item \(S > C\text{ }\rightarrow\)\textbf{HAVE}.
                            \item \(S < C\text{ }\rightarrow\)\textbf{WANT}.
                            \item \(S = C\text{ Kill somebody and assume }\rightarrow\)\textbf{HAVE}.
                        \end{enumerate}
                \end{enumerate}
        \end{enumerate}
    \item Remove \texttt{\_\_auth\_\_.tdp} from the \textbf{WANT} list (if it is there), issue a warning (using the dedicated command of \texttt{AUTHTR} and force a replace by \texttt{HAVE}.
\end{enumerate}

\section{Requirements for implementations}
\begin{enumerate}
    \item SSL encryption.
    \item Python + Twisted.
    \item Proper security measures: authentication, file permissions, network config etc.
    \item Communicate everything important to humans.
\end{enumerate}

\pagebreak
\section{Actions}
\label{sec:NA Actions}

The following actions exist:

\begin{table}[h!]
    \begin{tabular}{llll}
        \toprule
        \multicolumn{2}{c}{Action Name} \\
        \cmidrule(r){1-2}
        \textbf{Full} & \textbf{Internal} & \textbf{Description} \\
        \midrule
        Sync & \texttt{SYNC} & A standard sync operation. \\
        Live Network Access (\textsc{lna}) & \texttt{LNA} & A protocol for \emph{live}
        data access. \\
        User Modification & \texttt{USERMOD} & A user modification operation. \\
        \bottomrule
    \end{tabular}
    \caption{Actions.}
\end{table}

\subsection{Sync}
The Sync action is used to synchronize data between the server and a client.  Afterwards, the client can disconnect from the Internet, cut the cable in half and make a figurine of Chris Warrick.  If your cable is long enough.

\subsection{\textsc{lna}}

The \textsc{lna} action is used to interact with the Tiedot without storing the data anywhere, in a live connection fashion.  You cannot cut your cable and make a figurine of yours truly.  Unfortunately.

\subsection{User Modification}

The User Modification action provides access to exactly that: user modification.  Request are placed in a queue, and are fulfilled after issuing \texttt{QUIT}.

\pagebreak
\section{Commands}
\label{sec:NA Commands}

Note that this list may change in the future.  The following are valid for the version of the Tiedot distributed alongside this document (\version{}).

\begin{table}[h!]
    \begin{tabular}{p{2cm}p{4.9cm}|cc|p{6.2cm}}
        \toprule
        \textbf{Command} & \textbf{Arguments} & \textbf{C} & \textbf{S} &
        \textbf{Purpose} \\
        \midrule
        \texttt{HELLO} & last sync timestamp & + &   & Handshake. \\
        \texttt{HOWDY} & current server timestamp &   & + & Handshake. \\
        \texttt{AUTH} & username password& + &   & Authentication.  Password
        is sha512. \\
        \texttt{AUTHSTATUS} & auth status (0/1) &   & + & Authentication status.
        \texttt{\(\ominus\)BYE} is mandatory after a failed authentication. \\
        \texttt{ACTION} & action to use & + &   & Action to use.  More details in
        \fref{sec:NA Actions}. \\
        \texttt{ADDUSER} & username; password & + &   & Add an user. \\
        \texttt{DELUSER} & username; password & + &   & Delete an user. \\
        \texttt{CHANGEPWD} & old; new & + &   & Change current user's
        password. \\
        \texttt{USERMOD} & user modification status (0/1) &   & + & User
        modification status.  (\emph{Sync}, \emph{User Modification}: Immediately followed by
        mandatory \mbox{\texttt{AUTHTR}}, \texttt{FILES HAVE
        \mbox{\emph{\_\_auth\_\_.tdp}}} and \texttt{\(\ominus\)BYE}.) \\
        \texttt{AUTHTR} & &   & + & A warning informing that the
        \mbox{\texttt{\_\_auth\_\_.tdp}} file cannot be transferred due to
        security policies and a replacement of that file will be forced. \\
        \texttt{FILES} & \texttt{HAVE}/\texttt{WANT} (\emph{Sync}),
        \texttt{GET}/\texttt{PUT} (\emph{\textsc{lna}})
        \texttt{\emph{BP(files)}}& + & + & File transmission. \\
        \texttt{REFUSE} & refused item &  &  & The item is not accepted by the
        server. \\
        \texttt{QUIT} & reason (for humans to read) & + &   & Request to close the connection. \\
        \texttt{BYE} & \(\oplus\) current server TS &   & + & Connection
        closed. \\
        & \(\ominus\) \texttt{HELLO} TS & & & \\
        \bottomrule
    \end{tabular}
    \caption{Global NA commands.}
    \label{tab:Global NA Commands}
\end{table}

\begin{table}[h!]
    \begin{tabular}{p{2cm}p{4.9cm}|cc|p{6.2cm}}
        \toprule
        \textbf{Command} & \textbf{Arguments} & \textbf{C} & \textbf{S} &
        \textbf{Purpose} \\
        \midrule
        \texttt{DIRS} & existing directories &   & + & BP of a list of all
        existing directories. \\
        \texttt{CURDATA} & current data & + &   & Data Dict---client's data. \\
        \texttt{HAVE} & data to send to the client &   & + & Data
        Dict---client's outdated data. \\
        \texttt{WANT} & data to get from the client &   & + & Data
        Dict---server's outdated data. \\
        \texttt{DATAEQ} &  &   & + & Both sides have the same data. \\
        \bottomrule
    \end{tabular}
    \caption{Sync NA commands.}
    \label{tab:Sync Commands}
\end{table}

\begin{table}[h!]
    \begin{tabular}{p{2cm}p{4.9cm}|cc|p{6.2cm}}
        \toprule
        \textbf{Command} & \textbf{Arguments} & \textbf{C} & \textbf{S} &
        \textbf{Purpose} \\
        \midrule
        \texttt{TREE} & BP(existing files and dirs) &   & + & A tree of files
        on the server. \\
        \texttt{MKDIR} & path & + &   & Directories to create. \\
        \texttt{GET} & filename/\texttt{\_\_TREE\_\_} & + &   & Request a file
        to get. \\
        \texttt{PUT} & filename & + &   & Put a file up. \\
        \texttt{REPR} & filename & + &   & Get a \texttt{\_\_repr\_\_()} of the
        requested data, in plaintext. \\
        \bottomrule
    \end{tabular}
    \caption{\textsc{lna} commands.}
    \label{tab:LNA Commands}
\end{table}


%%% PART IV. Meta %%%


\part{Meta}
\label{chap:Meta}
\setcounter{section}{0}
\section{License}
Copyright \textcopyright{} 2012, Kwpolska.
All rights reserved.

Redistribution and use in source and binary forms, with or without
modification, are permitted provided that the following conditions are
met:

\begin{enumerate}
    \item Redistributions of source code must retain the above copyright
        notice, this list of conditions, and the following disclaimer.

    \item Redistributions in binary form must reproduce the above copyright
        notice, this list of conditions, and the following disclaimer in the
        documentation and/or other materials provided with the distribution.

    \item Neither the name of the author of this software nor the names of
        contributors to this software may be used to endorse or promote
        products derived from this software without specific prior written
        consent.
\end{enumerate}

THIS SOFTWARE IS PROVIDED BY THE COPYRIGHT HOLDERS AND CONTRIBUTORS
"AS IS" AND ANY EXPRESS OR IMPLIED WARRANTIES, INCLUDING, BUT NOT
LIMITED TO, THE IMPLIED WARRANTIES OF MERCHANTABILITY AND FITNESS FOR
A PARTICULAR PURPOSE ARE DISCLAIMED.  IN NO EVENT SHALL THE COPYRIGHT
OWNER OR CONTRIBUTORS BE LIABLE FOR ANY DIRECT, INDIRECT, INCIDENTAL,
SPECIAL, EXEMPLARY, OR CONSEQUENTIAL DAMAGES (INCLUDING, BUT NOT
LIMITED TO, PROCUREMENT OF SUBSTITUTE GOODS OR SERVICES; LOSS OF USE,
DATA, OR PROFITS; OR BUSINESS INTERRUPTION) HOWEVER CAUSED AND ON ANY
THEORY OF LIABILITY, WHETHER IN CONTRACT, STRICT LIABILITY, OR TORT
(INCLUDING NEGLIGENCE OR OTHERWISE) ARISING IN ANY WAY OUT OF THE USE
OF THIS SOFTWARE, EVEN IF ADVISED OF THE POSSIBILITY OF SUCH DAMAGE.
\pagebreak
\section{Revisions}
\begin{tabular}{lp{11.65cm}}
    \toprule
    \textbf{Timestamp} & \textbf{Changes} \\
    \midrule
    2012-03-28T19:00:00Z & initial version \\
    2012-03-29T14:00:00Z & [deleted] introduced \\
    2012-04-06T07:30:00Z & [deleted] introduced; values: [deleted] \\
    2012-04-18T19:00:00Z & paper document type, other style changes. \\
    2012-04-25T09:00:00Z & RDS can happen in 1000 years; reference updates.\\
    2012-06-11T18:14:00Z & Updating to reflect [deleted], [deleted] introduced, style modifications. \\
    2012-06-11T17:00:00Z & [deleted] introduced. \\
    2012-11-20T20:00:00Z & A very big update to reflect all the changes that were made in the recent days:

    \begin{enumerate}[(a)]
        \item a \emph{working} version of the RDS itself;
        \item better mechanisms;
        \item retirement of [deleted];
        \item making \textsc{rdsqt} much less important.
    \end{enumerate} \\
    2012-11-24T20:00:00Z & An update to cover the Sync mechanism. \\
    2012-11-25T20:00:00Z & \begin{enumerate}[(a)]
\item rename to KDS/Kw Data System (originally RDS/Rapid Data System);
\item more sync coverage;
\item retirement of \textsc{rdsqt}.
\end{enumerate} \\
2012-12-01T20:00:00Z & Sync updates, Py2k requirement. \\
2012-12-06T20:00:00Z & Even more sync updates. \\
2012-12-08T20:00:00Z & Rename to Tiedot, going up to git. \\
    \bottomrule
\end{tabular}
\end{document}
